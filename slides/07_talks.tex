\documentclass[aspectratio=169,12pt,t]{beamer}
\usepackage{graphicx}
\setbeameroption{hide notes}
\setbeamertemplate{note page}[plain]
\usepackage{listings}
\usepackage{ulem}

\input{LaTeX/header.tex}

%%%%%%%%%%%%%%%%%%%%%%%%%%%%%%%%%%%%%%%%%%%%%%%%%%%%%%%%%%%%%%%%%%%%%%
% end of header
%%%%%%%%%%%%%%%%%%%%%%%%%%%%%%%%%%%%%%%%%%%%%%%%%%%%%%%%%%%%%%%%%%%%%%

% title info
\title{Giving talks}
\subtitle{}
\author{\href{https://kbroman.org/BMI883}{\tt kbroman.org/BMI883} }
\institute{}
\date{\small \hspace{3in} Karl Broman \\
  \hspace{3in} \href{https://kbroman.org}{\color{foreground}
    \small \tt kbroman.org}}



\begin{document}



{
\setbeamertemplate{footline}{} % no page number here

\begin{frame}{Talks tips}

  \bi
\item It gets easier, but not easy
\item Know your audience
\item Don't go over time
\item Start with the application
\item Illustrations and figures $\gg$ words
\item Fully explain it, or leave it out
\item Think about what you like and dislike about others' talks
\item It's risky to mimic another's style
\item No need for an outline or ``thank you'' slide
  \bi
\item But I like a summary, particularly with pictures
  \ei
\item Have something to drink on hand
  \ei

\end{frame}
}



\end{document}

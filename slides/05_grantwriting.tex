\documentclass[aspectratio=169,12pt,t]{beamer}
\usepackage{graphicx}
\setbeameroption{hide notes}
\setbeamertemplate{note page}[plain]
\usepackage{listings}
\usepackage{ulem}

\input{LaTeX/header.tex}

%%%%%%%%%%%%%%%%%%%%%%%%%%%%%%%%%%%%%%%%%%%%%%%%%%%%%%%%%%%%%%%%%%%%%%
% end of header
%%%%%%%%%%%%%%%%%%%%%%%%%%%%%%%%%%%%%%%%%%%%%%%%%%%%%%%%%%%%%%%%%%%%%%

% title info
\title{Writing grants}
\subtitle{}
\author{\href{https://kbroman.org/BMI883}{\tt kbroman.org/BMI883} }
\institute{}
\date{\small \hspace{3in} Karl Broman \\
  \hspace{3in} \href{https://kbroman.org}{\color{foreground}
    \small \tt kbroman.org}}



\begin{document}



\begin{frame}{Grants}

  \bi
\item What problem are you going to solve?
\item Why is it important?
\item How are you going to do it?
\item How do we know it will work?
\item How will it be better than existing solutions?
  \ei

\end{frame}




\begin{frame}{Tips on grant writing}

\bi
\item Know your audience
\item Look at other grants
\item Have collaborators
\item Prepare early and get feedback
\item The idea is the important thing
  \bi
  \item Ideally wait until the idea is fully formed
  \ei
\item Preliminary data: showing you can do the work
\item Grants are reviewed by busy, middle-aged people
  \bi
  \item Not read front-to-back but in pieces
  \ei
\item Simple, clear language
\item Lots of white space
\item Provide text that reviewers can copy directly
\item Nice illustrations and figures
\ei

\end{frame}






\end{document}

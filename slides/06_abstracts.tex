\documentclass[aspectratio=169,12pt,t]{beamer}
\usepackage{graphicx}
\setbeameroption{hide notes}
\setbeamertemplate{note page}[plain]
\usepackage{listings}
\usepackage{ulem}

\input{LaTeX/header.tex}

%%%%%%%%%%%%%%%%%%%%%%%%%%%%%%%%%%%%%%%%%%%%%%%%%%%%%%%%%%%%%%%%%%%%%%
% end of header
%%%%%%%%%%%%%%%%%%%%%%%%%%%%%%%%%%%%%%%%%%%%%%%%%%%%%%%%%%%%%%%%%%%%%%

% title info
\title{Writing abstracts}
\subtitle{}
\author{\href{https://kbroman.org/BMI883}{\tt kbroman.org/BMI883} }
\institute{}
\date{\small \hspace{3in} Karl Broman \\
  \hspace{3in} \href{https://kbroman.org}{\color{foreground}
    \small \tt kbroman.org}}



\begin{document}



{
\setbeamertemplate{footline}{} % no page number here

\begin{frame}{Abstract tips}

  \bi
\item Know your audience
\item Know your purpose
  \bi
\item Paper: fully explain the contents (including the results)
\item Seminar: trying to get them to attend
\item Conference: convince them to let you talk
\item Grant: problem you'll solve; why it matters; what you'll do
   \ei
\item Focus on people outside your area
\item As with poetry, more likely to read if it's short
\item Save all of the abstracts you write
  \ei

\end{frame}
}



\end{document}

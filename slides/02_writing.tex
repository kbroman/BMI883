\documentclass[aspectratio=169,12pt,t]{beamer}
\usepackage{graphicx}
\setbeameroption{hide notes}
\setbeamertemplate{note page}[plain]
\usepackage{listings}
\usepackage{ulem}

\input{LaTeX/header.tex}

%%%%%%%%%%%%%%%%%%%%%%%%%%%%%%%%%%%%%%%%%%%%%%%%%%%%%%%%%%%%%%%%%%%%%%
% end of header
%%%%%%%%%%%%%%%%%%%%%%%%%%%%%%%%%%%%%%%%%%%%%%%%%%%%%%%%%%%%%%%%%%%%%%

% title info
\title{Writing papers}
\subtitle{}
\author{\href{https://kbroman.org/BMI883}{\tt kbroman.org/BMI883} }
\institute{}
\date{\small \hspace{3in} Karl Broman \\
  \hspace{3in} \href{https://kbroman.org}{\color{foreground}
    \small \tt kbroman.org}}



\begin{document}


% title slide
{
\setbeamertemplate{footline}{} % no page number here
\frame{
  \titlepage
} }




\begin{frame}{Tips}

      \bi
    \item Don't write it in order
      \bi
    \item Figures \& tables
    \item Results
    \item Methods
    \item Intro, discussion, abstract
      \ei
    \item Use simple language
      \bi
    \item Use the words you know
    \item If you can't think of how to say it, just skip it
      \ei
    \item Outline discussion points over time
    \item Don't try to be perfect
      \bi
    \item Get it on paper; edit later
      \ei
    \item Try speaking it, and writing down what you say
      \bi
      \item Like a talk, you're trying to tell a story and maintain
        the reader's interest
      \ei
    \item Supplemental tables/figures are great
    \ei


\end{frame}









\end{document}
